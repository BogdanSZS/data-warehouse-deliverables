\documentclass[aspectratio=169, 11pt]{beamer}

% Pachete
\usepackage[utf8]{inputenc}
\usepackage[T1]{fontenc}
\usepackage{textcomp}
\DeclareUnicodeCharacter{2009}{\,}  % tratează U+2009 ca o virgulă fină \,
\usepackage[romanian]{babel}
\usepackage{graphicx}
\usepackage{hyperref}
\usepackage{color}
\usepackage{ragged2e}
\usepackage{etoolbox}
\usepackage[numbers]{natbib}
\usepackage{listings}
\usepackage[absolute, overlay]{textpos}
\usepackage{amssymb}

% Definirea culorilor personalizate
\definecolor{darktext}{HTML}{064273}
\definecolor{lighttext}{HTML}{def3f6}
\definecolor{redtext}{HTML}{da1a32}
\definecolor{lightbackground}{HTML}{def3f6}
\definecolor{darkbackground}{HTML}{064273}
\definecolor{footerbackground}{HTML}{064273}
\definecolor{titlebackground}{HTML}{064273}
\definecolor{enumeratecolor}{HTML}{588b8b}

% Tema personalizată
\setbeamercolor{background canvas}{bg=lightbackground}
\setbeamercolor{normal text}{fg=darktext, bg=lightbackground}
\setbeamercolor{title}{fg=lighttext, bg=titlebackground}
\setbeamercolor{frametitle}{fg=lighttext, bg=darkbackground}
\setbeamercolor{footline}{bg=footerbackground, fg=lighttext}

% Culori pentru liste (itemize / enumerate)
\setbeamercolor{item}{fg=enumeratecolor}
\setbeamercolor{subitem}{fg=enumeratecolor}
\setbeamercolor{subsubitem}{fg=enumeratecolor}
\setbeamercolor{enumerate item}{fg=enumeratecolor}
\setbeamercolor{enumerate subitem}{fg=enumeratecolor}
\setbeamercolor{enumerate subsubitem}{fg=enumeratecolor}

% Customizare footer
\setbeamertemplate{footline}{
  \leavevmode%
  \hbox{%
  \begin{beamercolorbox}[wd=\paperwidth,ht=2.5ex,dp=1ex,leftskip=1em,rightskip=1em]{footline}%
    \usebeamerfont{footline}\insertshorttitle\hfill \insertshortdate
  \end{beamercolorbox}}%
}

% Eliminarea autorului și emailului din footer
\setbeamertemplate{navigation symbols}{}

% Customizare titlu
\setbeamertemplate{title page}{
  \vbox{}
  \vfill
  \begin{centering}
    \begin{beamercolorbox}[sep=8pt,center,rounded=true,shadow=true]{title}
      \usebeamerfont{title}\inserttitle\par
    \end{beamercolorbox}
    \vspace{0.5cm}
    \begin{flushleft}
      \usebeamerfont{author}\textbf{Autor:} Bogdan Szasz\\
      \textbf{Email:} Szasz.Te.Bogdan@student.utcluj.ro\\
      \textbf{Coordonator:} Șl. Dr. Ing. Călin Cenan\\
      \vspace{0.8em} \insertdate
    \end{flushleft}
  \end{centering}
  \vfill
}

% Tema prezentării
\usecolortheme{default}

% Setare globală pentru justificare
\AtBeginDocument{\let\raggedright\justifying}
% Also ensure lists & blocks are justified
\BeforeBeginEnvironment{itemize}{\justifying}
\BeforeBeginEnvironment{enumerate}{\justifying}
\BeforeBeginEnvironment{description}{\justifying}
\AtBeginEnvironment{frame}{\justifying}
\addtobeamertemplate{block begin}{}{\justifying}
\addtobeamertemplate{block alerted begin}{}{\justifying}
\addtobeamertemplate{block example begin}{}{\justifying}

% Titlu și autor
\title{Analiză de date în Looker Studio, folosind Superstore data source}
\date{\today}

\begin{document}

% Slide-ul de titlu
\begin{frame}
    \titlepage
\end{frame}

% Introducere
\begin{frame}{Introducere}
  \textbf{Context:}  
  Am realizat o analiză de date utilizând platforma \textbf{Google Looker Studio},
  pe baza unui set de date de tip retail denumit \textit{Superstore Dataset},
  conținând peste 10\,000 de înregistrări.  

  \vspace{0.5em}
  \textbf{Obiectiv:}  
  Construirea unui dashboard interactiv pentru monitorizarea indicatorilor de
  performanță (\textit{KPI}) — vânzări, profit, marjă și discount — împreună cu
  vizualizări geografice, segmentări de clienți și analize de produse.

  \vspace{0.5em}
  \textbf{Mediul de lucru:}  
  \begin{itemize}
    \item Sursa de date: Google Sheets (dataset importat dintr-un fișier Excel original).
    \item Platforma de raportare: Looker Studio.
    \item Număr total de câmpuri: 30 (inclusiv câmpuri derivate precum
    \textit{Margin \%}, \textit{Ship Delay}, \textit{Discount Tier}).
  \end{itemize}

  \vspace{0.5em}
  \footnotesize Acest raport oferă o perspectivă vizuală asupra performanței comerciale,
  permițând interpretarea rapidă a tendințelor și corelarea vânzărilor cu
  profitabilitatea.
\end{frame}

% Structura datasetului
\begin{frame}{Structura datasetului}
  \begin{columns}[T,onlytextwidth]
    % ---- Coloana din stânga ----
    \begin{column}{0.7\textwidth}
      \textbf{Descriere generală:}

      Setul de date conține informații retail din SUA între \textbf{2014-2017}:
      \begin{itemize}
        \item (\texttt{Order Date}, \texttt{Ship Date});
        \item (\texttt{Customer ID}, \texttt{Segment}, \texttt{Region});
        \item (\texttt{Category}, \texttt{Sub-Category}, \texttt{Product Name});
        \item (\texttt{Sales}, \texttt{Profit}, \texttt{Discount}, \texttt{Quantity});
      \end{itemize}

      \textbf{Câmpuri derivate adăugate:}
      \begin{itemize}
        \item \texttt{Margin \%} - raport între profit și vânzări;
        \item \texttt{Discount Tier} - clasificare a reducerilor;
        \item \texttt{Ship Delay (Days)} - diferența dintre livrare și comandă;
        \item \texttt{AOV (Avg Order Value)} - media valorii per comandă.
      \end{itemize}

    \end{column}

    % ---- Coloana din dreapta ----
    \begin{column}{0.28\textwidth}
      \centering
      \includegraphics[width=\linewidth]{figs/looker_fields.png}
      \vspace{0.3em}
      \footnotesize \textit{Figura:} Lista completă de câmpuri definite
    \end{column}
  \end{columns}
\end{frame}

% Dashboard principal — KPI-uri şi tendinţe generale
\begin{frame}{Dashboard principal — KPI-uri şi tendinţe generale}
  \begin{columns}[T,onlytextwidth]
    % ---- Stânga ----
    \begin{column}{0.5\textwidth}
      Panoul principal oferă o privire de ansamblu asupra performanţei generale:
      \vspace{0.5em}
      \begin{itemize}
        \item KPI-uri cheie: \textbf{Sales}, \textbf{Profit}, \textbf{Orders}, \textbf{Margin \%};
        \item Evoluţia lunară este afişată prin grafice de tip \textit{time series};
        \item Creştere constantă a volumului vânzărilor între 2014-2017;
        \item Uşoare variaţii sezoniere, mai ales în trimestrul 4.
      \end{itemize}
    \end{column}

    % ---- Dreapta ----
    \begin{column}{0.45\textwidth}
      \centering
      \includegraphics[width=\linewidth]{figs/looker_overview.png}
      \vspace{0.3em}
      \footnotesize \textit{Figura: Panou principal Looker Studio — evoluţia KPI-urilor în timp.}
    \end{column}
  \end{columns}
\end{frame}

% Analiza geografică a vânzărilor
\begin{frame}{Analiza geografică a vânzărilor}
  \begin{columns}[T,onlytextwidth]
    % ---- Stânga ----
    \begin{column}{0.5\textwidth}
      Vizualizarea geografică evidențiază principalele zone comerciale din SUA, 
      pe baza volumului total al vânzărilor.
      \vspace{0.5em}

      \begin{itemize}
        \item Cele mai mari concentrări sunt în \textbf{California}, \textbf{New York} și \textbf{Texas}.
        \item Punctele albastre mai intense indică valori de vânzări mai mari.
        \item Se observă o corelație între densitatea urbană și volumul de comenzi.
      \end{itemize}

      \vspace{0.5em}
      \small
      Indicatorul utilizat: \textbf{SUM(Sales)} agregat per \textit{City}.
    \end{column}

    % ---- Dreapta ----
    \begin{column}{0.45\textwidth}
      \centering
      \includegraphics[width=\linewidth]{figs/looker_geo.png}
      \vspace{0.3em}
      \footnotesize \textit{Figura: Distribuția geografică a vânzărilor în SUA.}
    \end{column}
  \end{columns}
\end{frame}

% Analiza pe segmente de clienţi şi profitabilitate
\begin{frame}{Analiza pe segmente de clienţi şi profitabilitate}
  \begin{columns}[T,onlytextwidth]
    % ---- Stânga ----
    \begin{column}{0.5\textwidth}
      \begin{itemize}
        \item \textbf{Consumer} generează cele mai multe vânzări totale;
        \item \textbf{Corporate} are o marjă mai ridicată şi profit stabil;
        \item \textbf{Home Office} are volum mai mic, dar discounturi mai mari;
        \item Profitul e corelat negativ cu nivelul mediu al discounturilor.
      \end{itemize}
      \vspace{0.5em}
      \small
      Câmpuri folosite: \texttt{Segment}, \texttt{SUM(Sales)}, \texttt{SUM(Profit)}, \texttt{AVG(Discount)}.
    \end{column}

    % ---- Dreapta ----
    \begin{column}{0.45\textwidth}
      \centering
      \includegraphics[width=\linewidth]{figs/looker_segments.png}
      \vspace{0.3em}
      \footnotesize \textit{Figura: Vânzări şi profit per segment de clienţi.}
    \end{column}
  \end{columns}
\end{frame}

% Analiza pe categorii şi subcategorii de produse
\begin{frame}{Analiza pe categorii şi subcategorii de produse}
  \begin{columns}[T,onlytextwidth]
    % ---- Stânga ----
    \begin{column}{0.55\textwidth}
      Această vizualizare compară performanţa principalelor categorii de produse
      din portofoliu: \textbf{Technology}, \textbf{Office Supplies} şi \textbf{Furniture}.
      \vspace{0.5em}

      \begin{itemize}
        \item \textbf{Technology} este categoria cu cea mai mare valoare totală a vânzărilor;
        \item \textbf{Office Supplies} are marje de profit stabile şi volum mare;
        \item \textbf{Furniture} prezintă variaţii mari între subcategorii (unele cu pierdere);
        \item Subcategoriile \textit{Phones} şi \textit{Chairs} conduc la nivel de profit absolut.
      \end{itemize}

      \vspace{0.5em}
      \small
      Câmpuri utilizate: \texttt{Category}, \texttt{Sub-Category}, \texttt{SUM(Sales)}, \texttt{SUM(Profit)}.
    \end{column}

    % ---- Dreapta ----
    \begin{column}{0.40\textwidth}
      \centering
      \includegraphics[width=\linewidth]{figs/looker_products.png}
      \vspace{0.3em}
      \footnotesize \textit{Figura: Vânzări şi profit per categorie şi subcategorie.}
    \end{column}
  \end{columns}
\end{frame}

% Produse cu cele mai mari profituri
\begin{frame}{Produse cu cele mai mari profituri}
  \begin{columns}[T,onlytextwidth]
    % ---- Stânga ----
    \begin{column}{0.5\textwidth}
      Tabelul evidenţiază cele mai profitabile produse,
      sortate descrescător după \textbf{Profit}, cu un \textit{heatmap} vizual.
      \vspace{0.7em}

      \textbf{Observaţii:}
      \begin{itemize}
        \item Categoria \textbf{Technology} domină topul (copiers, printers).
        \item Unele produse din \textbf{Furniture} au marje negative.
        \item \textbf{Office Supplies} are volum mare, dar profit variabil.
      \end{itemize}

      \vspace{0.5em}
      \footnotesize
      Câmpuri: \texttt{Category}, \texttt{Sub-Category}, \texttt{Product Name}, \texttt{Sales}, \texttt{Profit}, \texttt{Margin \%}.
    \end{column}

    % ---- Dreapta ----
    \begin{column}{0.45\textwidth}
      \centering
      \includegraphics[width=\linewidth]{figs/looker_heatmap.png}
      \vspace{0.3em}

      \footnotesize \textit{Figura: Tabel cu heatmap pe profituri individuale.}
    \end{column}
  \end{columns}
\end{frame}

% Clienţi principali după profit
\begin{frame}{Clienţi principali după profit}
  \begin{columns}[T,onlytextwidth]
    % ---- Stânga ----
    \begin{column}{0.5\textwidth}
      Graficul prezintă \textbf{top 10 clienţi} ordonaţi descrescător după profit total.
      \vspace{0.7em}

      \textbf{Observaţii:}
      \begin{itemize}
        \item Clienţii \textbf{Tamara Chand} şi \textbf{Raymond Buch} generează cele mai mari profituri.
        \item Diferenţa dintre primii doi şi restul este semnificativă.
        \item Profitul scade gradual, fără outlieri negativi.
      \end{itemize}

      \vspace{0.5em}
      \footnotesize
      Câmpuri utilizate: \texttt{Customer Name}, \texttt{Profit}.
    \end{column}

    % ---- Dreapta ----
    \begin{column}{0.45\textwidth}
      \centering
      \includegraphics[width=\linewidth]{figs/looker_customers.png}
      \vspace{0.3em}
      \footnotesize \textit{Figura: Top 10 clienţi ordonaţi descrescător după profit.}
    \end{column}
  \end{columns}
\end{frame}

% Discounturi şi marje
\begin{frame}{Discounturi şi marje}
  \begin{columns}[T,onlytextwidth]
    % ---- Stânga ----
    \begin{column}{0.53\textwidth}
      Graficele arată impactul nivelului de discount asupra
      \textbf{vânzărilor} şi \textbf{marjei de profit}.
      \vspace{0.7em}

      \textbf{Observaţii:}
      \begin{itemize}
        \item Cele mai mari vânzări provin din categoria \textbf{No Discount}.
        \item Discounturile foarte mari (\textbf{40\%+}) duc la marje negative.
        \item Există o corelaţie inversă între volum şi profitabilitate.
      \end{itemize}

      \vspace{0.5em}
      \footnotesize
      Câmpuri utilizate: \texttt{Discount Tier}, \texttt{Sales}, \texttt{Margin \%}.
    \end{column}

    % ---- Dreapta ----
    \begin{column}{0.45\textwidth}
      \centering
      \includegraphics[width=\linewidth]{figs/looker_discounts.png}
      \vspace{0.3em}
      \footnotesize \textit{Figura: Vânzări şi marje medii pe intervale de discount.}
    \end{column}
  \end{columns}
\end{frame}

% Relaţia vânzări-profit pe categorii
\begin{frame}{Relaţia vânzări-profit pe categorii}
  \begin{columns}[T,onlytextwidth]
    % ---- Stânga ----
    \begin{column}{0.5\textwidth}
      Diagrama de tip \textbf{bubble chart} arată relaţia dintre
      \textbf{vânzări} şi \textbf{profit} pentru fiecare categorie principală.
      \vspace{0.7em}

      \textbf{Observaţii:}
      \begin{itemize}
        \item \textbf{Technology} are cele mai mari vânzări şi profit.
        \item \textbf{Office Supplies} are volum mare, dar marjă mai modestă.
        \item \textbf{Furniture} are profit semnificativ mai mic, indicând costuri ridicate.
      \end{itemize}

      \vspace{0.5em}
      \footnotesize
      Câmpuri utilizate: \texttt{Category}, \texttt{Sales}, \texttt{Profit}.
    \end{column}

    % ---- Dreapta ----
    \begin{column}{0.45\textwidth}
      \centering
      \includegraphics[width=\linewidth]{figs/looker_scatter.png}
      \vspace{0.3em}
      \footnotesize \textit{Figura: Raport vânzări-profit la nivel de categorie.}
    \end{column}
  \end{columns}
\end{frame}


% Concluzii
\begin{frame}{Concluzii}
  \begin{itemize}
    \item Google Looker Studio permite crearea rapidă de rapoarte interactive, bazate pe date brute din Google Sheets.
    \item Analiza a evidențiat corelații clare între discounturi, vânzări și profit.
    \item Vizualizările oferă o perspectivă intuitivă asupra performanței categoriilor și clienților.
    \item Instrumentul este potrivit pentru raportări descriptive și monitorizarea indicatorilor de business.
  \end{itemize}
\end{frame}

% Slide: Mulțumiri
\begin{frame}
    \centering
    \vfill
    {\LARGE \textbf{Vă mulțumesc pentru atenție!}}
    \vfill
\end{frame}

\end{document}